\begin{abstract}
Liquid-fueled Molten Salt Reactor (MSR) systems represent advances in safety, economics, sustainability, and proliferation-resistance. Therefore, \gls{MSR} has been selected as one of the promising reactors by the \gls{GIF}. Basically, the \gls{MSR} has been designed to operate based on Th/$^{233}$U fuel cycle. Since $^{233}$U does not exist in nature, it is required to examine available fissile materials to replace the $^{233}$U in the startup fuel. Here, five different types of initial fissile materials are proposed for transitioning to thorium fuel cycle in the \gls{SD-TMSR}. Plutonium mixed with \gls{LEU} (19.79\%), \gls{LEU} (19.79\%), Plutonium reactor-grade, \gls{TRU} from LWR spent fuel (SF) and finally $^{233}$U for comparison purpose are investigated. In the present paper, two different feed mechanisms are applied. Consequently, the multiplication factor, inventories of important nuclides and net production of $^{233}$U are studied. Moreover, the molten salt Temperature Coefficient of Reactivity $(\alpha$$_{T}$) is negative for startup and equilibrium states. The results show that the continuous flow of Pu reactor-grade helps in transition to thorium fuel cycle within a relatively short time ($\approx$ $4.5$ $years$) compared to $26$ $years$ for Th/$^{233}$U startup fuel. Meanwhile, using \gls{TRU} as initial fissile materials shows the possibility of operating the SD-TMSR for a long period of time ($\approx$ $40$ $years$) without any external feed of $^{233}$U. 
\end{abstract}

%The SD-TMSR (2,250 MW$_{th}$) is a Single-fluid Double-zone Thorium-based Molten Salt Reactor. The active core of the SD-TMSR is divided into the inner zone (486 fuel tubes) and the outer zone (522 fuel tubes) to improve the Th-U breeding performance. The lack of computational reactor analysis software that deal with MSR cores prevents improvement. This work adopted the SERPENT-2 Monte Carlo code to analyze the whole core model of the SD-TMSR. Built-in SERPENT-2 capabilities simulated online reprocessing and refueling and calculated the multiplication factor and Breeding Ratio (BR). We found that the molten salt Temperature Coefficient of Reactivity $(\alpha$$_{T}$) was negative for initial and equilibrium states. This study investigated the variation of the multiplication factor, BR, and build-up of important nuclides in the core as a function of burn-up time. Under online reprocessing and refueling, we studied the variation of the reactivity during 60 years of reactor operation. The present work has introduced a feasible reprocessing scheme. In addition, it has investigated the neutron flux energy spectrum at initial and equilibrium state.
%In conclusion, the full core of SD-TMSR was modeled and the net production of $^{233}$U was found to be positive.
