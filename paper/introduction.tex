\section{Introduction}
The \gls{GIF} has defined eight technology goals for the next generation nuclear systems. These goals have been defined in four broad areas: safety and reliability, economics, sustainability, non-proliferation and physical protection \cite{doe2002technology}. \glspl{MSR} have many advantages that consistent with \gls{GIF}'s goals, for example, liquid fuel, inherent safety, online reprocessing and refueling, excellent neutron economy and operation under ambient pressure \cite{siemer2015molten,rosenthal1970molten}. Therefore, in 2002 the \gls{MSR} has been chosen as one of the promising reactors by this forum \cite{doe2002technology,pioro2016handbook}.
In the \gls{MSR}, the fuel supposed to be in the form of liquid dissolved in molten salt (e.g., LiF or NaCl). This liquid fuel salt (e.g., LiF-BeF$_2$-ThF$_4$-$^{233}$UF$_4$) constantly circulates through the core and allows transferring fission heat. 
 
The Single-fluid Double-zone Thorium-based Molten Salt Reactor (SD-TMSR-2,250 MWth) was introduced by the \gls{CAS} \cite{li_optimization_2018}. The SD-TMSR is a graphite-moderated thermal-spectrum MSR. In the SD-TMSR the fissile and fertile elements are integrated into the same salt. In addition, the active core is divided into two zones, the radius of the fuel channels in the outer zone is modified to be larger than the radius of the fuel channels in the inner zone to improve the breeding ratio \cite{nuttin2005potential,li_optimization_2018}.

Basically, the \gls{MSR} has been designed to apply the Th/$^{233}$U fuel cycle \cite{rykhlevskii2019modeling,nuttin2005potential,merle2004scenarios,rosenthal1970molten}. Hence, the fertile isotope $^{232}$Th is converted to the fissile isotope $^{233}$U, an isotope that is not exist in nature. Therefore it is required to examine available fissile materials (e.g., $^{235}$U and Pu) to replace the $^{233}$U in the startup fuel \cite{betzler2016modeling,zou2018transition}. The thorium fuel cycle transition can be achieved after reaching the doubling time\footnote{Time required to produce enough amount of $^{233}$U to trigger a new SD-TMSR.} of $^{233}$U.

Betzler, et al. discussed the simulation of the start-up of a MSBR unit cell with LEU (19.79\%) and Pu from \gls{LWR} spent fuel (SF) as initial fissile materials \cite{betzler2016modeling}. They concluded that the plutonium vector extracted from LWR SF serves as the best alternative source to $^{233}$U thanks to the highest ratio of fissile isotopes \cite{betzler2016modeling}. Zou, et al. introduced two approaches for the thorium fuel cycle transition in \gls{TMSR}: in-core transition and ex-core transition. In the former way, the TMSR is launched with existing fissile material and thorium as a fertile material, then the bred $^{233}$U from thorium is rerouted into the core for criticality. In contrast, the latter way tends to store the bred $^{233}$U out of the core until there is enough amount to start a new TMSR \cite{zou2018transition}.
Meanwhile, Zou, et al. studied the transitioning to thorium fuel cycle in a small modular Th-based molten salt reactor (smTMSR) using TRUs as startup fuel. They concluded that the transition to thorium fuel cycle can be achieved in thermal smTMSR with a proper fuel fraction \cite{zou2018preliminary}.
Heuer, et al., discussed the transition characteristics of the \gls{MSFR} under different launching scenarios (e.g., enriched uranium and TRU), they concluded that starting the Thorium fuel cycle is feasible while closing the current fuel cycle and adopting stockpile incineration in MSRs for optimizing the long-term waste management \cite{heuer2014towards}.

Indeed, there are various researches that revolve around starting the MSRs with fissile materials alternative to $^{233}$U. Many of these researches focus on the fast-spectrum MSRs \cite{ashraf2019modeling,ashraf2018nuclear,heuer2014towards,fiorina2013investigation}, while little focus on thermal-spectrum MSRs \cite{betzler2016modeling,zou2018preliminary,zou2018transition}. Nevertheless, starting the \gls{SD-TMSR} with other fissile materials (except $^{233}$U) not found in the literature. Therefore, the main object of the present paper is to discuss the simulation of the operation of \gls{SD-TMSR} for a long period of time (60 years) with different initial fissile materials and without any external feed of $^{233}$U to achieve the thorium fuel cycle transition. To do that, we investigate five types of initial fissile materials based on \gls{LEU}, Pu mixed with \gls{LEU}, Pu reactor-grade, \gls{TRU} from LWR SF and $^{233}$U \cite{de2000scenarios}. Moreover, two different feed mechanisms are used as follows:

\begin{itemize}
	\item Continuous feed flow of thorium from Th stockpile and $^{233}$U from \texttt{Pa-decay tank}\footnote{An imaginary tank used to store protactinium extracted from the core.}, where the removal rate of $^{233}$Pa = feed rate of $^{233}$U. \cite{betzler2016modeling}.
	\item Continuous injection of Heavy Metal (HM) (excluding Th) and simultaneously feed of all or part of produced $^{233}$U from \texttt{Pa-decay tank}.
\end{itemize}

All calculations presented in the present paper were performed using SERPENT-2 version 2.1.30. We used the MSR burnup routine provided by SERPENT-2 to simulate continuous online reprocessing and refueling. SERPENT-2 uses an internal calculation routine for solving the set of Bateman equations describing the changes in the material compositions caused by neutron-induced reactions and radioactive decay \cite{leppanen2014serpent}. Additionally, SERPENT-2 allows us to conduct the burnup calculations on computer clusters with multiple cores using distributed-memory MPI parallelization.

This present paper is organized as follows: after an introduction about \gls{MSR} systems, the model description is discussed in section 2. Methodology and tools is descried in section 3. Extraction and feed mechanisms are addressed in section 4. Section 5 focuses on the results and discussion. Finally, section 6 highlights the conclusions.
