\section{Introduction} \label{Introduction}
The \gls{GIF} defines eight technological goals for the next generation
nuclear systems. These goals are defined in four broad areas: safety and reliability, non-proliferation and physical protection, economics, and sustainability \cite{doe2002technology}. The \gls{MSR} has many advantages that agree
with \gls{GIF}'s goals, like: liquid fuel, inherent safety, online
reprocessing and refueling, excellent neutron economy, and operation near
atmospheric
pressure in a primary loop \cite{siemer2015molten,rosenthal1970molten}.
Thus, the \gls{GIF} selected the \gls{MSR} as one of the promising Generation-IV
reactors \cite{doe2002technology,pioro2016handbook}.
In \glspl{MSR}, the fuel is dissolved in a molten salt (e.g., LiF or NaCl) \cite{betzler_impacts_2019};
this liquid fuel salt (e.g., LiF-BeF$_2$-ThF$_4$-$^{233}$UF$_4$) constantly
circulates through the core and allows fission heat to transfer from the reactor
core to intermediate heat exchangers.

The Single-fluid Double-zone Thorium-based Molten Salt Reactor (SD-TMSR) with a thermal power of 2,250
MW$_{th}$ was proposed for the first time by ORNL as early as in the 1960s, which was called Molten salt breeding reactor (MSBR) \cite{robertson_conceptual_1971}.
The SD-TMSR is a graphite-moderated thermal-spectrum \gls{MSR} operating in Th/$^{233}$U
fuel cycle. In the SD-TMSR, the fissile and fertile elements are integrated
into a single salt. To improve the breeding ratio, the active core is divided into two zones:
the radius of the fuel channels in the outer zone is modified to be larger
than the radius of the fuel channels in the inner zone \cite{li_optimization_2018}.

Historically, the thermal-spectrum \gls{MSR} was designed for the Th/$^{233}$U
fuel cycle \cite{rykhlevskii2019modeling,nuttin2005potential,
	merle2004scenarios,rosenthal1970molten}. This design assumes that we have
fissile $^{233}$U inventory to start up new \glspl{MSR}. However, $^{233}$U does not
exist in the Earth's crust and can only be produced from fertile $^{232}$Th in 
specific nuclear facilities. Therefore, we examine alternative fissile
materials to replace the $^{233}$U in the startup fuel
composition \cite{betzler2016modeling,zou2018transition}. The thorium fuel
cycle transition can be achieved after reaching the doubling
time of $^{233}$U (required time to produce enough $^{233}$U to start up a new unit) because in this case, all startup fissile material
is being substituted by newly produced $^{233}$U.

Betzler \emph{et al.} (2016) discussed the simulation of the startup of a \gls{MSBR} unit
cell with \gls{LEU} (19.79\%) and Pu from \gls{LWR} spent nuclear fuel (SNF) as initial
fissile materials \cite{betzler2016modeling}. They concluded that the 
Pu vector extracted from LWR SNF is the best alternative source to 
$^{233}$U because it has a high ratio of fissile isotopes
\cite{betzler2016modeling}. Zou \emph{et al.} (2018) introduced two approaches for
the thorium fuel cycle transition in the \gls{TMSR}: (1) in-core transition and 
(2) ex-core transition. In the first approach, the TMSR is launched with 
existing fissile material and thorium as a fertile material; then the 
$^{233}$U bred from thorium is rerouted into the core to maintain criticality. 
In contrast, the second approach tends to store produced $^{233}$U out of the 
core until there is enough to start a new TMSR \cite{zou2018transition}.

Zou \emph{et al.} (2018) studied the transitioning to thorium fuel
cycle in a small modular Thorium-based Molten Salt Reactor (smTMSR) using \gls{TRU}
as startup fuel. They concluded that the transition to a thorium fuel cycle can
be achieved in a thermal smTMSR with a proper fuel fraction 
\cite{zou2018preliminary}.

Heuer \emph{et al.} (2014) discussed the transition characteristics of the \gls{MSFR}
under different launching scenarios (e.g., enriched uranium and TRU); they concluded that starting the thorium fuel cycle is feasible in the MSFR while closing the current fuel cycle and optimizing the management of the long-term wastes \cite{heuer2014towards}.

Cui \emph{et al.} analyzed the fuel transition from enriched $^{235}$U/Th and Pu/Th to $^{233}$U/Th for the MSBR-like design. Cui \emph{et al.} found that the fuel transition can be achieved by using: (1) enriched uranium with greater than 40\% enrichment, which rises the proliferation concerns; (2) Pu from LWR (burn-up of 60 GWd/t) spent nuclear fuel \cite{cui2017transition,cui2018possible}.

The research effort described in \cite{cui2017transition} and
\cite{cui2018possible} is most similar to the work
presented in this paper. However, a few major differences
are: (1) we used built-in truly continuous fuel depletion capabilities in
SERPENT-2 \cite{aufiero2013extended} while Cui \emph{et al.} employed an in-house tool named MSR
Reprocessing Sequence (MSR-RS) \cite{ZOU2015114} which couples with SCALE and employs a batch-wise approach (the burnup simulation stops at a given time and restarts with a new liquid fuel composition after removal of discarded materials and addition of fissile/fertile materials); (2) we investigated different initial loading cases (HALEU, Pu+HALEU, reactor-grade Pu, TRU and $^{233}$U) while Cui \emph{et al.} studied only uranium and Pu as the startup fissile materials; (3) we adopted the SD-TMSR core geometry optimized by Li \emph{et al.} \cite{li_optimization_2018} which is different from the MSBR-like model adopted by Cui \emph{et al.}; (4) Cui \emph{et al.} studied the high-enriched uranium (e$>$90) while we excluded it due to the unavailability of this material for use as an initial reactor fuel (non-proliferation issues) and because the performance of this material is expected to be similar to $^{233}$U; (5) Cui \emph{et al.} introduced two scenarios for thorium fuel cycle transition: a Breeding and Burning (B\&B) scenario and a Pre-breeding scenario while we adopting different approach by introducing two different feed mechanisms: thorium and non-thorium feed mechanism.

Comparing with Cui \emph{et al.}, we used different geometry, volumes, densities, and fuel salt compositions. In Cui's scenarios, the amount of Th is kept constant. $^{233}$U is fed only in (B\&B) scenario and extra fissile materials would have to be added into the core if the produced $^{233}$U is not enough to maintain the reactor criticality. In the Pre-breeding scenario, all the $^{233}$U produced from the decay of the extracted $^{233}$Pa stored outside the core until it reaches the required startup mass for a new reactor. Thus, the criticality of the core is maintained by refueling external fissile material. 
In current work, using the thorium feed mechanism, we simultaneously feed Th and all or part of produced $^{233}$U from the \texttt{Pa-decay tank} into the core. The excess $^{233}$U (if exist) is stored outside the core. In the non-thorium feed mechanism, we continuously inject external heavy metals (HALEU, Pu mixed with HALEU, reactor-grade Pu, or TRU) and part of produced $^{233}$U from the \texttt{Pa-decay tank}. Here we feed a part of produced $^{233}$U because in some cases (e.g., TRU) refueling only external heavy metals failed to maintain the core criticality (limitation is dm$_{total fuel}$ $\leq$ 0.1\%). In the non-thorium feed mechanism, the amount of Th decreases during reactor operation while it is constant in the thorium feed mechanism.

Various previous works explore starting the
\glspl{MSR} with fissile materials alternative to $^{233}$U. Many such publications have focused on the fast-spectrum \glspl{MSR} \cite{heuer2014towards,ashraf2019modeling,
	ashraf2018nuclear, rykhlevskii_fuel_2019, betzler_impacts_2019,
	fiorina2013investigation}, while few focus on
thermal-spectrum \glspl{MSR} \cite{betzler2016modeling,
	zou2018transition,zou2018preliminary}.
The main objective of the present paper is to establish feasible strategies for thorium fuel cycle transition in the SD-TMSR with
various initial fissile materials and without any external feed of $^{233}$U. We investigate five different
initial fissile materials: \gls{HALEU}, Pu mixed with \gls{HALEU}, reactor-grade Pu \footnote{Reactor-grade Pu is a mixture of Pu isotopes chemically extracted from PWR SNF with 33 GWd/tHM burnup} \cite{marka1993explosive}, \gls{TRU} from LWR SNF \cite{de2000scenarios}, and $^{233}$U. Two different feed mechanisms were selected:

\begin{itemize}
	\item \textbf{Thorium feed mechanism}: continuous feed flow of thorium from the external stockpile and $^{233}$U from the \texttt{Pa-decay tank}\footnote{An external tank used to store protactinium extracted from the core.}, where the removal rate of $^{233}$Pa = feed rate of $^{233}$U \cite{betzler2016modeling}.
	\item \textbf{Non-thorium feed mechanism}: continuous injection of external heavy metals (\gls{HALEU}, Pu mixed with \gls{HALEU}, reactor-grade Pu, \gls{TRU}) and simultaneous feed of all or fraction of $^{233}$U from the \texttt{Pa-decay tank}.
\end{itemize}

$^{233}$U is contaminated with $^{232}$U, which produced from parasitic (n,2n) reactions in $^{233}$Pa, or in $^{232}$Th, or in $^{233}$U itself. $^{208}$Tl a daughter of $^{232}$U emits intense $\gamma$-radiation and this makes $^{233}$U undesirable for nuclear weapons. Moreover, for nonproliferation reasons, bred $^{233}$U could be diluted with $^{238}$U to produce denatured fuel, which not suitable for nuclear weapons \cite{dolan2017molten}.

All calculations presented in this paper are performed using SERPENT-2 version 2.1.31\footnote{SERPENT-2 is a 3D continuous energy Monte Carlo neutron 
	transport and burnup code.} \cite{leppanen2014serpent}. We use the MSR burnup routine provided by SERPENT-2 to simulate continuous online reprocessing and refueling. SERPENT-2 uses an internal calculation routine for solving the Bateman equations describing the changes in the material compositions caused by neutron-induced reactions and radioactive decay \cite{leppanen2014serpent}. Additionally, SERPENT-2 enables burnup calculations on computer clusters with multiple cores using distributed-memory MPI parallelization.

This paper is organized as follows: section \ref{Model-description} discusses the model description, section \ref{Methodology-and-tools} describes methodology and tools, section \ref{Feed-and-extraction-rates} addresses extraction and feed mechanisms, section \ref{Results-and-discussion} focuses on the results and discussion, and section \ref{Conclusion} highlights the conclusions.