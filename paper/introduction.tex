\section{Introduction}
The \gls{GIF} has defined eight technology goals for the next generation nuclear systems. These goals have been defined in four broad areas: safety and reliability, economics, sustainability, non-proliferation and physical protection \cite{doe2002technology}. \glspl{MSR} have many advantages that consistent with \gls{GIF}'s goals, for example, liquid fuel, inherent safety, online reprocessing and refueling, excellent neutron economy and operation under ambient pressure \cite{siemer2015molten,rosenthal1970molten}. Therefore, in 2002 the \gls{MSR} has been chosen as one of the promising reactors by this forum \cite{doe2002technology,pioro2016handbook}.
In the \gls{MSR}, the fuel supposed to be in the form of liquid dissolved in molten salt (e.g., LiF or NaCl). This liquid fuel salt (e.g., LiF-BeF$_2$-ThF$_4$-$^{233}$UF$_4$) constantly circulates through the core and allows transferring fission heat. 
 
The Single-fluid Double-zone Thorium-based Molten Salt Reactor (SD-TMSR-2,250 MWth) was introduced by the \gls{CAS} \cite{li_optimization_2018}. The SD-TMSR is a graphite-moderated thermal-spectrum MSR. In the SD-TMSR the fissile and fertile elements are integrated into the same salt. In addition, the active core is divided into two zones, the radius of the fuel channels in the outer zone is modified to be larger than the radius of the fuel channels in the inner zone to improve the breeding ratio \cite{nuttin2005potential,li_optimization_2018}.

Basically, the \gls{MSR} has been designed to apply the Th/$^{233}$U fuel cycle \cite{rykhlevskii2019modeling,nuttin2005potential,merle2004scenarios,rosenthal1970molten}. Hence, the fertile isotope $^{232}$Th is converted to the fissile isotope $^{233}$U, an isotope that is not exist in nature. Therefore it is required to examine available fissile materials (e.g., $^{235}$U and Pu) to replace the $^{233}$U in the startup fuel \cite{betzler2016modeling,zou2018transition}. The thorium fuel cycle transition can be achieved after reaching the doubling time\footnote{Time required to produce enough amount of $^{233}$U to trigger a new SD-TMSR.} of $^{233}$U.

Betzler, et al. discussed the simulation of the start-up of a MSBR unit cell with LEU (19.79\%) and Pu from \gls{LWR} spent fuel (SF) as initial fissile materials \cite{betzler2016modeling}. They concluded that the plutonium vector extracted from LWR SF serves as the best alternative source to $^{233}$U thanks to the highest ratio of fissile isotopes \cite{betzler2016modeling}. Zou, et al. introduced two approaches for the thorium fuel cycle transition in \gls{TMSR}: in-core transition and ex-core transition. In the former way, the TMSR is launched with existing fissile material and thorium as a fertile material, then the bred $^{233}$U from thorium is rerouted into the core for criticality. In contrast, the latter way tends to store the bred $^{233}$U out of the core until there is enough amount to start a new TMSR \cite{zou2018transition}.
Meanwhile, Zou, et al. studied the transitioning to thorium fuel cycle in a small modular Th-based molten salt reactor (smTMSR) using TRUs as startup fuel. They concluded that the transition to thorium fuel cycle can be achieved in thermal smTMSR with a proper fuel fraction \cite{zou2018preliminary}.
Heuer, et al., discussed the transition characteristics of the \gls{MSFR} under different launching scenarios (e.g., enriched uranium and TRU), they concluded that starting the Thorium fuel cycle is feasible while closing the current fuel cycle and adopting stockpile incineration in MSRs for optimizing the long-term waste management \cite{heuer2014towards}.

Indeed, there are various researches that revolve around starting the MSRs with fissile materials alternative to $^{233}$U. Many of these researches focus on the fast-spectrum MSRs \cite{ashraf2019modeling,ashraf2018nuclear,heuer2014towards,fiorina2013investigation}, while little focus on thermal-spectrum MSRs \cite{betzler2016modeling,zou2018preliminary,zou2018transition}. Nevertheless, starting the \gls{SD-TMSR} with other fissile materials (except $^{233}$U) not found in the literature. Therefore, the main object of the present paper is to discuss the simulation of the operation of \gls{SD-TMSR} for a long period of time (60 years) with different initial fissile materials and without any external feed of $^{233}$U to achieve the thorium fuel cycle transition. To do that, we investigate five types of initial fissile materials based on \gls{LEU}, Pu mixed with \gls{LEU}, Pu reactor-grade, \gls{TRU} from LWR SF and $^{233}$U \cite{de2000scenarios}. Moreover, two different feed mechanisms are used as follows:

\begin{itemize}
	\item Continuous feed flow of thorium from Th stockpile and $^{233}$U from \texttt{Pa-decay tank}\footnote{An imaginary tank used to store protactinium extracted from the core.}, where the removal rate of $^{233}$Pa = feed rate of $^{233}$U. \cite{betzler2016modeling}.
	\item Continuous injection of Heavy Metal (HM) (excluding Th) and simultaneously feed of all or part of produced $^{233}$U from \texttt{Pa-decay tank}.
\end{itemize}

All calculations presented in the present paper were performed using SERPENT-2 version 2.1.30. We used the MSR burnup routine provided by SERPENT-2 to simulate continuous online reprocessing and refueling. SERPENT-2 uses an internal calculation routine for solving the set of Bateman equations describing the changes in the material compositions caused by neutron-induced reactions and radioactive decay \cite{leppanen2014serpent}. Additionally, SERPENT-2 allows us to conduct the burnup calculations on computer clusters with multiple cores using distributed-memory MPI parallelization.

This present paper is organized as follows: after an introduction about \gls{MSR} systems, the model description is discussed in section 2. Methodology and tools is descried in section 3. Extraction and feed mechanisms are addressed in section 4. Section 5 focuses on the results and discussion. Finally, section 6 highlights the conclusions.

%produce enough amount of $^{233}$U to start a new \gls{SD-TMSR}. 
%Therefore, the aim of this paper is to operate the \gls{SD-TMSR} for a long period of time with different initial fissile materials without external feed of $^{233}$U (that does not exist anywhere) to produce enough amount of $^{233}$U that is required to start a new \gls{SD-TMSR}. 


%The \gls{MSR} is the only liquid-fueled reactor that the \gls{GIF} has chosen as one of the six promising concepts \cite{doe2002technology,pioro2016handbook}. The reactor's liquid fuel distinguishes it from other nuclear reactors. \glspl{MSR} can operate extended time without shutdown for refueling with a superior neutron economy \cite{doe2002technology}. Introducing online reprocessing and refueling systems would ensure sustainability. In a \gls{MSR}, liquid fuel salt (e.g. a mixture of LiF-BeF$_2$-ThF$_4$-$^{233}$UF$_4$) circulates through the core and transports fission heat to the \glspl{IHX}. Finally, this type of reactor has a high conversion efficiency and a negative temperature coefficient.

%Both thermal and fast \gls{MSR} concepts exist in the literature \cite{pioro2016handbook,engel1979development}. For thermal \glspl{MSR}, hexagonal graphite prisms act as moderators. In contrast, fast spectrum \glspl{MSR} lack moderators. A \gls{MSR} can breed in both cases, making them optimal for realizing a desirable Thorium-Uranium fuel cycle \cite{hargraves2010liquid}.

%In the 1950s, \gls{ORNL} performed experiments that provided a basis for \gls{MSR} feasibility. In 1958, a 5 MW$_{th}$ homogeneous reactor experiment called HRE-2 utilized water-based liquid fuel to illustrate the intrinsic stability of homogeneous reactors. In the 1960s, \gls{ORNL} performed the \gls{MSRE} \cite{engel1979development,haubenreich1970experience,fiorinibasis,robertson1965msre}. These experiments demonstrated the capability of circulating a liquid fluoride mixture without significant corrosion problems. Researchers used a nickel-based alloy (Hastelloy N) as structural material and controlled the fuel's oxidation by using a U$^{3+}$/U$^{4+}$ buffer. The experiment also tested the continuous reprocessing of liquid fuel. The results showed that introducing a noble gas system (i.e. gas bubbling system) during the fuel cycle helps extract gaseous fission products \cite{pioro2016handbook}.

%\gls{ORNL} later researched the \gls{MSBR} \cite{rosenthal1970molten}. There were two types of \gls{MSBR}: the single-fluid molten salt reactor and the two-fluid molten salt reactor. While the single-fluid MSR dissolves fertile and fissile materials in the same salt, the two-fluid MSR physically separates them. The two-fluid MSR had a higher breeding performance than the single-fluid MSR. However, the single-fluid MSR can operate as an iso-breeding reactor as long as there is an efficient reprocessing system for the fuel salt \cite{rosenthal1970molten}.

%Despite the success of the \gls{MSRE}, \gls{ORNL} halted work on the \gls{MSBR} in 1976. The \gls{Euratom} later revived the project. To enhance the breeding performance of the \gls{MSBR}, \gls{Euratom} introduced the idea of dividing the core into multiple zones of fuel channels with different radii \cite{nuttin2005potential}. The results showed that increasing the radii of the outer fuel channels relative to the inner radii improved the reactor's breeding and allowed it to out-perform the reactor with a single-zone core and large fuel channels \cite{nuttin2005potential}.

%Later, fast spectrum MSRs were introduced. Several fast MSR designs include: \gls{MSFR}, \gls{EVOL} \cite{serp2014molten}, \gls{MOSART} \cite{boussier2012molten}, REBUS-3700 \cite{mourogov2006potentialities} and Molten Chloride Salt Fast Reactor \cite{taube1978fast}. \gls{GIF} has elected to further research \gls{MSFR} and \gls{MOSART} \cite{serp2014molten,boussier2012molten,locatelli2013generation,pettersen2016coupled}. Both designs feature the fast-spectrum, Th-based fuel, and liquid fuel circulation. The \gls{MSFR} concept initially gained momentum at \gls{CNRS} in France, and the \gls{MOSART} concept is in progress in Russia. 

%Indeed, several challenges interfere with the commercial adoption of \gls{MSFR}. For example, large initial inventory of $^{233}$U and relatively long doubling time for $^{233}$U production. Moreover, material scientists must identify or create structural materials which have reasonable lifespan in extreme operational conditions (high temperature, large neutron flux, chemically aggressive salt). Finally, breeding of high-quiality fissile material in the \gls{MSFR} blanket rises significant nuclear non-proliferation concerns \cite{pettersen2016coupled,merle2009minimizing}. 

%The thermal MSR has priority due to the low inventory of initial fissile $^{233}$U and short doubling time for its production \cite{ZOU2015114}. 
%Nagy et al. studied the effect of core zoning on the breeding performance and graphite lifespan of a graphite-moderated MSR \cite{nagy2011new,nagy2012effects}. Li et al. divided the active core into two zones to improve the breeding performance \cite{li_optimization_2018}. The central zone is moderated, while the unmoderated outer zone acts as a subcritical fast spectrum zone \cite{li_optimization_2018}.

%In 2011, the \gls{CAS} established the strategic project ``Future Advanced Nuclear Energy $-$ Thorium-based Molten Salt Reactor System (TMSR)". Since then, several studies aimed to improve the \gls{TMSR} to efficiently utilize the Th-U fuel cycle \cite{li_optimization_2018,jiang2012advanced,li2015analysis,li2017model}.
%Separately, we can highlight the issue of a closed Th-U fuel cycle in the thermal spectrum. A number of studies have shown the fundamental possibility of organizing such a cycle in a heavy-water reactor and referred to the limitation of the burn-up of solid fuels due to the accumulation of the fission products (FPs) \cite{dastur1995thorium,bergel2004mode,bergelson2008op}. The question of feasibility and parameters of a closed U-Th fuel cycle in a \gls{TMSR} reactor is relevant.

%Liquid-fueled systems require specific neutron transport and depletion tools that capture online fuel reprocessing and refueling. Specifically, the circulation of the liquid fuel and the continuous feed or removal of elements (e.g. fissile material injection into and Fission Products (FPs) extraction from the fuel salt) are out-of-scope for most contemporary nuclear reactor physics software, which was originally written primarily to simulate solid-fueled reactors. The SERPENT-2 \cite{leppanen2014serpent} code extension takes into account the online fuel reprocessing and its effects on depletion calculations \cite{aufiero2013extended}. 

%The present paper aims to analyze the fuel depletion of the \gls{SD-TMSR} during 60 years reactor operation time using SERPENT-2. This work investigates the inventory of isotopes in the core to establish the advantages of the \gls{TMSR} concept applied to energy production (based on the thorium fuel cycle). All calculations presented in the present work were implemented using SERPENT-2 version 2.1.30. We adopted the built-in SERPENT-2 subroutine to simulate continuous online reprocessing and refueling in contrast with other batch-wise codes \footnote{In the batch-wise technique, the simulation stops at a certain time and restarts after removing of poisoning isotopes and addition of fissile and/or fertile materials.} (e.g. SaltProc \cite{rykhlevskii2019modeling} and MCNP6-PYTHON \cite{Jeong2016}). 
%SERPENT-2 allowed us to conduct burn-up calculations on computer clusters with multiple cores using distributed-memory MPI parallelization. The drift of the delayed neutron precursors is not examined in this paper.

%The present paper is organized as follows: after a general introduction about \gls{MSR} systems, section 2 briefly presents the \gls{SD-TMSR}. Section 3 focuses on the methodology and calculation tools used in this work. Fission products extraction modeling is described in section 4. Section 5 highlights the results and discussion. Finally, section 6 summarizes the conclusions.
=======
=======
>>>>>>> 0a3da8d52a196f4b660758727d5c99e3afedf170
The \gls{GIF} has defined eight technology goals for the next generation
nuclear systems. These goals are: safety and reliability, economics,
sustainability, non-proliferation and physical protection
\cite{doe2002technology}. The \gls{MSR} has many advantages that consistent
with \gls{GIF}'s goals, for example, liquid fuel, inherent safety, online
reprocessing and refueling, excellent neutron economy and operation near
atmospheric
pressure in a primary loop \cite{siemer2015molten,rosenthal1970molten}.
Thus, the \gls{GIF} selected \gls{MSR} as one of the promising Generation-IV
reactors \cite{doe2002technology,pioro2016handbook}.
In the \gls{MSR}, the fuel is dissolved in a molten salt (e.g., LiF or NaCl).
This liquid fuel salt (e.g., LiF-BeF$_2$-ThF$_4$-$^{233}$UF$_4$) constantly
circulates through the core and allows transferring fission heat from reactor
core to heat exchanger.

The Single-fluid Double-zone Thorium-based Molten Salt Reactor (SD-TMSR-2,250
MW$_{th}$) was introduced by the \gls{CAS} \cite{li_optimization_2018}. The
SD-TMSR
is a graphite-moderated thermal-spectrum \gls{MSR} operating in Th/$^{233}$U
fuel cycle. In the SD-TMSR the fissile and fertile elements are integrated
into the same salt. In addition, the active core is divided into two zones,
the radius of the fuel channels in the outer zone is modified to be larger
than the radius of the fuel channels in the inner zone to improve the breeding
ratio \cite{nuttin2005potential,li_optimization_2018}.

Historically, the thermal-spectrum \gls{MSR} was designed for the Th/$^{233}$U
fuel cycle \cite{rykhlevskii2019modeling,nuttin2005potential,
merle2004scenarios,rosenthal1970molten}. This design assumes that we have
fissile $^{233}$U inventory to startup new \glspl{MSR}. But $^{233}$U does not
exist in the Earth's crust and can be produced from fertile $^{232}$Th only in
the nuclear reactor. Therefore, it is required to examine alternative fissile
materials (e.g., $^{235}$U) to replace the $^{233}$U in the startup fuel
composition \cite{betzler2016modeling,zou2018transition}. The thorium fuel
cycle transition can be achieved after reaching the doubling
time\footnote{Time required to produce enough amount of $^{233}$U to trigger a
new SD-TMSR.} of $^{233}$U because in this case all startup fissile material
is being substituted by newly produced $^{233}$U.

Betzler \emph{et al.} discussed the simulation of the startup of a MSBR unit
cell with \gls{LEU} (19.79\%) and Pu from \gls{LWR} spent fuel (SF) as initial
fissile materials \cite{betzler2016modeling}. They concluded that the 
plutonium vector extracted from LWR SF is the best alternative source to 
$^{233}$U because it has the highest ratio of fissile isotopes
\cite{betzler2016modeling}. Zou \emph{et al.} introduced two approaches for
the thorium fuel cycle transition in \gls{TMSR}: (1) in-core transition and 
(2) ex-core transition. In the first approach, the TMSR is launched with 
existing fissile material and thorium as a fertile material; then the 
$^{233}$U bred from thorium is rerouted into the core to maintain criticality. 
In contrast, the second approache tends to store produced $^{233}$U out of the 
core until there is enough amount to start a new TMSR \cite{zou2018transition}.
Additionally, Zou \emph{et al.} studied the transitioning to thorium fuel
cycle in a small modular Th-based molten salt reactor (smTMSR) using \gls{TRU}
as startup fuel. They concluded that the transition to thorium fuel cycle can
be achieved in thermal smTMSR with a proper fuel fraction 
\cite{zou2018preliminary}.

Heuer \emph{et al.} discussed the transition characteristics of the \gls{MSFR}
under different launching scenarios (e.g., enriched uranium and TRU)
\cite{heuer2014towards}.

Indeed, there are various researches that revolve around starting the
\glspl{MSR} with fissile materials alternative to $^{233}$U. Many of these
researches focus on the fast-spectrum \glspl{MSR} \cite{ashraf2019modeling,
ashraf2018nuclear, rykhlevskii_fuel_2019, betzler_impacts_2019,
heuer2014towards,fiorina2013investigation}, while little focus on
thermal-spectrum \glspl{MSR} \cite{betzler2016modeling, zou2018preliminary,
zou2018transition}. Nevertheless, starting the \gls{SD-TMSR} with other
fissile materials (except $^{233}$U) was not studied before. Therefore,
the main object of the present paper is to discuss the simulation of the
\gls{SD-TMSR} operation for a lifetime-long period of time (60 years) with
different initial fissile materials and without any external feed of $^{233}$U
to achieve the thorium fuel cycle transition. We investigated five different
initial fissile materials: \gls{LEU}, Pu, and \gls{TRU} from LWR SF
\cite{de2000scenarios}. Moreover, two different feed scenarios were selected:
\begin{itemize}
	\item Continuous feed flow of thorium and $^{233}$U from \texttt{Pa-decay 
	tank}, where the removal rate of $^{233}$Pa = feed rate of $^{233}$U. 
	\cite{betzler2016modeling}.
	\item Continuous feed flow of (Heavy Metal (HM) except for Th) + all or 
	part of $^{233}$U from \texttt{Pa-decay tank}.
\end{itemize}
<<<<<<< HEAD
>>>>>>> upstream/master
=======
>>>>>>> 0a3da8d52a196f4b660758727d5c99e3afedf170

This present paper is organized as follows: after an introduction about 
\gls{MSR} systems, the model description is discussed in section 2. 
Methodology and tools is descried in section 3. Extraction and feed mechanisms 
are addressed in section 4. Section 5 focuses on the results and discussion. 
Finally, section 6 highlights the conclusions.

%produce enough amount of $^{233}$U to start a new \gls{SD-TMSR}.
%Therefore, the aim of this paper is to operate the \gls{SD-TMSR} for a long 
%period of time with different initial fissile materials without external feed 
%of $^{233}$U (that does not exist anywhere) to produce enough amount of 
%$^{233}$U that is required to start a new \gls{SD-TMSR}.


%The \gls{MSR} is the only liquid-fueled reactor that the \gls{GIF} has chosen 
%as one of the six promising concepts 
%\cite{doe2002technology,pioro2016handbook}. The reactor's liquid fuel 
%distinguishes it from other nuclear reactors. \glspl{MSR} can operate 
%extended 
%time without shutdown for refueling with a superior neutron economy 
%\cite{doe2002technology}. Introducing online reprocessing and refueling 
%systems would ensure sustainability. In a \gls{MSR}, liquid fuel salt (e.g. a 
%mixture of LiF-BeF$_2$-ThF$_4$-$^{233}$UF$_4$) circulates through the core 
%and 
%transports fission heat to the \glspl{IHX}. Finally, this type of reactor has 
%a high conversion efficiency and a negative temperature coefficient.

%Both thermal and fast \gls{MSR} concepts exist in the literature 
%\cite{pioro2016handbook,engel1979development}. For thermal \glspl{MSR}, 
%hexagonal graphite prisms act as moderators. In contrast, fast spectrum 
%\glspl{MSR} lack moderators. A \gls{MSR} can breed in both cases, making them 
%optimal for realizing a desirable Thorium-Uranium fuel cycle 
%\cite{hargraves2010liquid}.

%In the 1950s, \gls{ORNL} performed experiments that provided a basis for 
%\gls{MSR} feasibility. In 1958, a 5 MW$_{th}$ homogeneous reactor experiment 
%called HRE-2 utilized water-based liquid fuel to illustrate the intrinsic 
%stability of homogeneous reactors. In the 1960s, \gls{ORNL} performed the 
%\gls{MSRE} 
%\cite{engel1979development,haubenreich1970experience,fiorinibasis,robertson1965msre}.
% These experiments demonstrated the capability of circulating a liquid 
%fluoride mixture without significant corrosion problems. Researchers used a 
%nickel-based alloy (Hastelloy N) as structural material and controlled the 
%fuel's oxidation by using a U$^{3+}$/U$^{4+}$ buffer. The experiment also 
%tested the continuous reprocessing of liquid fuel. The results showed that 
%introducing a noble gas system (i.e. gas bubbling system) during the fuel 
%cycle helps extract gaseous fission products \cite{pioro2016handbook}.

%\gls{ORNL} later researched the \gls{MSBR} \cite{rosenthal1970molten}. There 
%were two types of \gls{MSBR}: the single-fluid molten salt reactor and the 
%two-fluid molten salt reactor. While the single-fluid MSR dissolves fertile 
%and fissile materials in the same salt, the two-fluid MSR physically 
%separates 
%them. The two-fluid MSR had a higher breeding performance than the 
%single-fluid MSR. However, the single-fluid MSR can operate as an 
%iso-breeding 
%reactor as long as there is an efficient reprocessing system for the fuel 
%salt 
%\cite{rosenthal1970molten}.

%Despite the success of the \gls{MSRE}, \gls{ORNL} halted work on the 
%\gls{MSBR} in 1976. The \gls{Euratom} later revived the project. To enhance 
%the breeding performance of the \gls{MSBR}, \gls{Euratom} introduced the idea 
%of dividing the core into multiple zones of fuel channels with different 
%radii 
%\cite{nuttin2005potential}. The results showed that increasing the radii of 
%the outer fuel channels relative to the inner radii improved the reactor's 
%breeding and allowed it to out-perform the reactor with a single-zone core 
%and 
%large fuel channels \cite{nuttin2005potential}.

%Later, fast spectrum MSRs were introduced. Several fast MSR designs include: 
%\gls{MSFR}, \gls{EVOL} \cite{serp2014molten}, \gls{MOSART} 
%\cite{boussier2012molten}, REBUS-3700 \cite{mourogov2006potentialities} and 
%Molten Chloride Salt Fast Reactor \cite{taube1978fast}. \gls{GIF} has elected 
%to further research \gls{MSFR} and \gls{MOSART} 
%\cite{serp2014molten,boussier2012molten,locatelli2013generation,pettersen2016coupled}.
% Both designs feature the fast-spectrum, Th-based fuel, and liquid fuel 
%circulation. The \gls{MSFR} concept initially gained momentum at \gls{CNRS} 
%in 
%France, and the \gls{MOSART} concept is in progress in Russia.

%Indeed, several challenges interfere with the commercial adoption of 
%\gls{MSFR}. For example, large initial inventory of $^{233}$U and relatively 
%long doubling time for $^{233}$U production. Moreover, material scientists 
%must identify or create structural materials which have reasonable lifespan 
%in 
%extreme operational conditions (high temperature, large neutron flux, 
%chemically aggressive salt). Finally, breeding of high-quiality fissile 
%material in the \gls{MSFR} blanket rises significant nuclear 
%non-proliferation 
%concerns \cite{pettersen2016coupled,merle2009minimizing}.

%The thermal MSR has priority due to the low inventory of initial fissile 
%$^{233}$U and short doubling time for its production \cite{ZOU2015114}.
%Nagy et al. studied the effect of core zoning on the breeding performance and 
%graphite lifespan of a graphite-moderated MSR 
%\cite{nagy2011new,nagy2012effects}. Li et al. divided the active core into 
%two 
%zones to improve the breeding performance \cite{li_optimization_2018}. The 
%central zone is moderated, while the unmoderated outer zone acts as a 
%subcritical fast spectrum zone \cite{li_optimization_2018}.

%In 2011, the \gls{CAS} established the strategic project ``Future Advanced 
%Nuclear Energy $-$ Thorium-based Molten Salt Reactor System (TMSR)". Since 
%then, several studies aimed to improve the \gls{TMSR} to efficiently utilize 
%the Th-U fuel cycle 
%\cite{li_optimization_2018,jiang2012advanced,li2015analysis,li2017model}.
%Separately, we can highlight the issue of a closed Th-U fuel cycle in the 
%thermal spectrum. A number of studies have shown the fundamental possibility 
%of organizing such a cycle in a heavy-water reactor and referred to the 
%limitation of the burn-up of solid fuels due to the accumulation of the 
%fission products (FPs) 
%\cite{dastur1995thorium,bergel2004mode,bergelson2008op}. The question of 
%feasibility and parameters of a closed U-Th fuel cycle in a \gls{TMSR} 
%reactor 
%is relevant.

%Liquid-fueled systems require specific neutron transport and depletion tools 
%that capture online fuel reprocessing and refueling. Specifically, the 
%circulation of the liquid fuel and the continuous feed or removal of elements 
%(e.g. fissile material injection into and Fission Products (FPs) extraction 
%from the fuel salt) are out-of-scope for most contemporary nuclear reactor 
%physics software, which was originally written primarily to simulate 
%solid-fueled reactors. The SERPENT-2 \cite{leppanen2014serpent} code 
%extension 
%takes into account the online fuel reprocessing and its effects on depletion 
%calculations \cite{aufiero2013extended}.

%The present paper aims to analyze the fuel depletion of the \gls{SD-TMSR} 
%during 60 years reactor operation time using SERPENT-2. This work 
%investigates 
%the inventory of isotopes in the core to establish the advantages of the 
%\gls{TMSR} concept applied to energy production (based on the thorium fuel 
%cycle). All calculations presented in the present work were implemented using 
%SERPENT-2 version 2.1.30. We adopted the built-in SERPENT-2 subroutine to 
%simulate continuous online reprocessing and refueling in contrast with other 
%batch-wise codes \footnote{In the batch-wise technique, the simulation stops 
%at a certain time and restarts after removing of poisoning isotopes and 
%addition of fissile and/or fertile materials.} (e.g. SaltProc 
%\cite{rykhlevskii2019modeling} and MCNP6-PYTHON \cite{Jeong2016}).
%SERPENT-2 allowed us to conduct burn-up calculations on computer clusters 
%with multiple cores using distributed-memory MPI parallelization. The drift 
%of 
%the delayed neutron precursors is not examined in this paper.

%The present paper is organized as follows: after a general introduction about 
%\gls{MSR} systems, section 2 briefly presents the \gls{SD-TMSR}. Section 3 
%focuses on the methodology and calculation tools used in this work. Fission 
%products extraction modeling is described in section 4. Section 5 highlights 
%the results and discussion. Finally, section 6 summarizes the conclusions.