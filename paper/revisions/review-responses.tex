
%
% Copyright 2007, 2008, 2009 Elsevier Ltd
%
% This file is part of the 'Elsarticle Bundle'.
% ---------------------------------------------
%
% It may be distributed under the conditions of the LaTeX Project Public
% License, either version 1.2 of this license or (at your option) any
% later version.  The latest version of this license is in
%    http://www.latex-project.org/lppl.txt
% and version 1.2 or later is part of all distributions of LaTeX
% version 1999/12/01 or later.
%
% The list of all files belonging to the 'Elsarticle Bundle' is
% given in the file `manifest.txt'.
%

% Template article for Elsevier's document class `elsarticle'
% with numbered style bibliographic references
% SP 2008/03/01
%
%
%
% $Id: elsarticle-template-num.tex 4 2009-10-24 08:22:58Z rishi $
%
%
%\documentclass[preprint,12pt]{elsarticle}
\documentclass[answers,11pt]{exam}

% \documentclass[preprint,review,12pt]{elsarticle}

% Use the options 1p,twocolumn; 3p; 3p,twocolumn; 5p; or 5p,twocolumn
% for a journal layout:
% \documentclass[final,1p,times]{elsarticle}
% \documentclass[final,1p,times,twocolumn]{elsarticle}
% \documentclass[final,3p,times]{elsarticle}
% \documentclass[final,3p,times,twocolumn]{elsarticle}
% \documentclass[final,5p,times]{elsarticle}
% \documentclass[final,5p,times,twocolumn]{elsarticle}

% if you use PostScript figures in your article
% use the graphics package for simple commands
% \usepackage{graphics}
% or use the graphicx package for more complicated commands
\usepackage{graphicx}
% or use the epsfig package if you prefer to use the old commands
% \usepackage{epsfig}

% The amssymb package provides various useful mathematical symbols
\usepackage{amssymb}
% The amsthm package provides extended theorem environments
% \usepackage{amsthm}
\usepackage{amsmath}

% The lineno packages adds line numbers. Start line numbering with
% \begin{linenumbers}, end it with \end{linenumbers}. Or switch it on
% for the whole article with \linenumbers after \end{frontmatter}.
\usepackage{lineno}

% I like to be in control
\usepackage{placeins}

% natbib.sty is loaded by default. However, natbib options can be
% provided with \biboptions{...} command. Following options are
% valid:

%   round  -  round parentheses are used (default)
%   square -  square brackets are used   [option]
%   curly  -  curly braces are used      {option}
%   angle  -  angle brackets are used    <option>
%   semicolon  -  multiple citations separated by semi-colon
%   colon  - same as semicolon, an earlier confusion
%   comma  -  separated by comma
%   numbers-  selects numerical citations
%   super  -  numerical citations as superscripts
%   sort   -  sorts multiple citations according to order in ref. list
%   sort&compress   -  like sort, but also compresses numerical citations
%   compress - compresses without sorting
%
% \biboptions{comma,round}

% \biboptions{}


% Katy Huff addtions
\usepackage{xspace}
\usepackage{color}

\usepackage{multirow}
\usepackage[hyphens]{url}


\usepackage[acronym,toc]{glossaries}
\newacronym{<++>}{<++>}{<++>}
\newacronym[longplural={metric tons of heavy metal}]{MTHM}{MTHM}{metric ton of heavy metal}
\newacronym{ABM}{ABM}{agent-based modeling}
\newacronym{ACDIS}{ACDIS}{Program in Arms Control \& Domestic and International Security}
\newacronym{AHTR}{AHTR}{Advanced High Temperature Reactor}
\newacronym{ANDRA}{ANDRA}{Agence Nationale pour la gestion des D\'echets RAdioactifs, the French National Agency for Radioactive Waste Management}
\newacronym{ANL}{ANL}{Argonne National Laboratory}
\newacronym{ANS}{ANS}{American Nuclear Society}
\newacronym{API}{API}{application programming interface}
\newacronym{ARE}{ARE}{Aircraft Reactor Experiment}
\newacronym{ARFC}{ARFC}{Advanced Reactors and Fuel Cycles}
\newacronym{ASME}{ASME}{American Society of Mechanical Engineers}
\newacronym{ATWS}{ATWS}{Anticipated Transient Without Scram}
\newacronym{BDBE}{BDBE}{Beyond Design Basis Event}
\newacronym{BIDS}{BIDS}{Berkeley Institute for Data Science}
\newacronym{BR}{BR}{Breeding Ratio}
\newacronym{CAFCA}{CAFCA}{ Code for Advanced Fuel Cycles Assessment }
\newacronym{CAS}{CAS}{Chinese Academy of Sciences} 
\newacronym{CDTN}{CDTN}{Centro de Desenvolvimento da Tecnologia Nuclear}
\newacronym{CEA}{CEA}{Commissariat \`a l'\'Energie Atomique et aux \'Energies Alternatives}
\newacronym{CFD}{CFD}{Computational Fluid Dynamics}
\newacronym{CI}{CI}{continuous integration}
\newacronym{CNEN}{CNEN}{Comiss\~{a}o Nacional de Energia Nuclear}
\newacronym{CNERG}{CNERG}{Computational Nuclear Engineering Research Group}
\newacronym{COSI}{COSI}{Commelini-Sicard}
\newacronym{COTS}{COTS}{commercial, off-the-shelf}
\newacronym{CSNF}{CSNF}{commercial spent nuclear fuel}
\newacronym{CTAH}{CTAHs}{Coiled Tube Air Heaters}
\newacronym{CUBIT}{CUBIT}{CUBIT Geometry and Mesh Generation Toolkit}
\newacronym{CURIE}{CURIE}{Centralized Used Fuel Resource for Information Exchange}
\newacronym{DAG}{DAG}{directed acyclic graph}
\newacronym{DANESS}{DANESS}{Dynamic Analysis of Nuclear Energy System Strategies}
\newacronym{DBE}{DBE}{Design Basis Event}
\newacronym{DESAE}{DESAE}{Dynamic Analysis of Nuclear Energy Systems Strategies}
\newacronym{DHS}{DHS}{Department of Homeland Security}
\newacronym{DOE}{DOE}{Department of Energy}
\newacronym{DRACS}{DRACS}{Direct Reactor Auxiliary Cooling System}
\newacronym{DRE}{DRE}{dynamic resource exchange}
\newacronym{DSNF}{DSNF}{DOE spent nuclear fuel}
\newacronym{DYMOND}{DYMOND}{Dynamic Model of Nuclear Development }
\newacronym{EBS}{EBS}{Engineered Barrier System}
\newacronym{EDF}{EDF}{Électricité de France}
\newacronym{EDZ}{EDZ}{Excavation Disturbed Zone}
\newacronym{EFPY}{EFPY}{Effective Full-Power Years}
\newacronym{EIA}{EIA}{U.S. Energy Information Administration}
\newacronym{EPA}{EPA}{Environmental Protection Agency}
\newacronym{EPR}{EPR}{European Pressurized Reactors}
\newacronym{EP}{EP}{Engineering Physics}
\newacronym{EU}{EU}{European Union}
\newacronym{FCO}{FCO}{Fuel Cycle Options}
\newacronym{FCT}{FCT}{Fuel Cycle Technology}
\newacronym{FEHM}{FEHM}{Finite Element Heat and Mass Transfer}
\newacronym{FEPs}{FEPs}{Features, Events, and Processes}
\newacronym{FHR}{FHR}{Fluoride-Salt-Cooled High-Temperature Reactor}
\newacronym{FLiBe}{FLiBe}{Fluoride-Lithium-Beryllium}
\newacronym{FP}{FP}{Fission Product}
\newacronym{FTC}{FTC}{Fuel Temperature Coefficient}
\newacronym{GDSE}{GDSE}{Generic Disposal System Environment}
\newacronym{GDSM}{GDSM}{Generic Disposal System Model}
\newacronym{GENIUSv1}{GENIUSv1}{Global Evaluation of Nuclear Infrastructure Utilization Scenarios, Version 1}
\newacronym{GENIUSv2}{GENIUSv2}{Global Evaluation of Nuclear Infrastructure Utilization Scenarios, Version 2}
\newacronym{GENIUS}{GENIUS}{Global Evaluation of Nuclear Infrastructure Utilization Scenarios}
\newacronym{GIF}{GIF}{Generation IV International Forum}
\newacronym{GPAM}{GPAM}{Generic Performance Assessment Model}
\newacronym{GRSAC}{GRSAC}{Graphite Reactor Severe Accident Code}
\newacronym{GUI}{GUI}{graphical user interface}
\newacronym{HLW}{HLW}{high level waste}
\newacronym{HPC}{HPC}{high-performance computing}
\newacronym{HTC}{HTC}{high-throughput computing}
\newacronym{HTGR}{HTGR}{High Temperature Gas-Cooled Reactor}
\newacronym{IAEA}{IAEA}{International Atomic Energy Agency}
\newacronym{IEMA}{IEMA}{Illinois Emergency Mangament Agency}
\newacronym{IHLRWM}{IHLRWM}{International High Level Radioactive Waste Management}
\newacronym{INL}{INL}{Idaho National Laboratory}
\newacronym{IHX}{IHX}{Intermediate Heat Exchanger}
\newacronym{IPRR1}{IRP-R1}{Instituto de Pesquisas Radioativas Reator 1}
\newacronym{IRP}{IRP}{Integrated Research Project}
\newacronym{ISFSI}{ISFSI}{Independent Spent Fuel Storage Installation}
\newacronym{ISRG}{ISRG}{Independent Student Research Group}
\newacronym{JFNK}{JFNK}{Jacobian-Free Newton Krylov}
\newacronym{LANL}{LANL}{Los Alamos National Laboratory}
\newacronym{LBNL}{LBNL}{Lawrence Berkeley National Laboratory}
\newacronym{LCOE}{LCOE}{levelized cost of electricity}
\newacronym{LDRD}{LDRD}{laboratory directed research and development}
\newacronym{LFR}{LFR}{Lead-Cooled Fast Reactor}
\newacronym{LLNL}{LLNL}{Lawrence Livermore National Laboratory}
\newacronym{LMFBR}{LMFBR}{Liquid Metal Fast Breeder Reactor}
\newacronym{LOFC}{LOFC}{Loss of Forced Cooling}
\newacronym{LOHS}{LOHS}{Loss of Heat Sink}
\newacronym{LOLA}{LOLA}{Loss of Large Area}
\newacronym{LP}{LP}{linear program}
\newacronym{LWR}{LWR}{Light Water Reactor}
\newacronym{MAGNOX}{MAGNOX}{Magnesium Alloy Graphie Moderated Gas Cooled Uranium Oxide Reactor}
\newacronym{MA}{MA}{minor actinide}
\newacronym{MCNP}{MCNP}{Monte Carlo N-Particle code}
\newacronym{MILP}{MILP}{mixed-integer linear program}
\newacronym{MIT}{MIT}{the Massachusetts Institute of Technology}
\newacronym{MOAB}{MOAB}{Mesh-Oriented datABase}
\newacronym{MOOSE}{MOOSE}{Multiphysics Object-Oriented Simulation Environment}
\newacronym{MOSART}{MOSART}{Molten Salt Actinide Recycler and Transmuter}
\newacronym{MOX}{MOX}{mixed oxide}
\newacronym{MPI}{MPI}{Message Passing Interface}
\newacronym{MRPP}{MRPP}{Multiregion Processing Plant}
\newacronym{MSBR}{MSBR}{Molten Salt Breeder Reactor}
\newacronym{MSFR}{MSFR}{Molten Salt Fast Reactor}
\newacronym{MSRE}{MSRE}{Molten Salt Reactor Experiment}
\newacronym{MSR}{MSR}{Molten Salt Reactor}
\newacronym{MTC}{MTC}{Moderator Temperature Coefficient}
\newacronym{NAGRA}{NAGRA}{National Cooperative for the Disposal of Radioactive Waste}
\newacronym{NEAMS}{NEAMS}{Nuclear Engineering Advanced Modeling and Simulation}
\newacronym{NEUP}{NEUP}{Nuclear Energy University Programs}
\newacronym{NFCSim}{NFCSim}{Nuclear Fuel Cycle Simulator}
\newacronym{NGNP}{NGNP}{Next Generation Nuclear Plant}
\newacronym{NMWPC}{NMWPC}{Nuclear MW Per Capita}
\newacronym{NNSA}{NNSA}{National Nuclear Security Administration}
\newacronym{NPP}{NPP}{Nuclear Power Plant}
\newacronym{NPRE}{NPRE}{Department of Nuclear, Plasma, and Radiological Engineering}
\newacronym{NQA1}{NQA-1}{Nuclear Quality Assurance - 1}
\newacronym{NRC}{NRC}{Nuclear Regulatory Commission}
\newacronym{NSF}{NSF}{National Science Foundation}
\newacronym{NSSC}{NSSC}{Nuclear Science and Security Consortium}
\newacronym{NUWASTE}{NUWASTE}{Nuclear Waste Assessment System for Technical Evaluation}
\newacronym{NWF}{NWF}{Nuclear Waste Fund}
\newacronym{NWTRB}{NWTRB}{Nuclear Waste Technical Review Board}
\newacronym{OCRWM}{OCRWM}{Office of Civilian Radioactive Waste Management}
\newacronym{ORION}{ORION}{ORION}
\newacronym{ORNL}{ORNL}{Oak Ridge National Laboratory}
\newacronym{PARCS}{PARCS}{Purdue Advanced Reactor Core Simulator}
\newacronym{PBAHTR}{PB-AHTR}{Pebble Bed Advanced High Temperature Reactor}
\newacronym{PBFHR}{PB-FHR}{Pebble-Bed Fluoride-Salt-Cooled High-Temperature Reactor}
\newacronym{PEI}{PEI}{Peak Environmental Impact}
\newacronym{PH}{PRONGHORN}{PRONGHORN}
\newacronym{PRIS}{PRIS}{Power Reactor Information System}
\newacronym{PRKE}{PRKE}{Point Reactor Kinetics Equations}
\newacronym{PSPG}{PSPG}{Pressure-Stabilizing/Petrov-Galerkin}
\newacronym{PWAR}{PWAR}{Pratt and Whitney Aircraft Reactor}
\newacronym{PWR}{PWR}{Pressurized Water Reactor}
\newacronym{PyNE}{PyNE}{Python toolkit for Nuclear Engineering}
\newacronym{PyRK}{PyRK}{Python for Reactor Kinetics}
\newacronym{QA}{QA}{quality assurance}
\newacronym{RDD}{RD\&D}{Research Development and Demonstration}
\newacronym{RD}{R\&D}{Research and Development}
\newacronym{REE}{REE}{rare earth element}
\newacronym{RELAP}{RELAP}{Reactor Excursion and Leak Analysis Program}
\newacronym{RIA}{RIA}{Reactivity Insertion Accident}
\newacronym{RIF}{RIF}{Region-Institution-Facility}
\newacronym{ROD}{ROD}{Reactor Optimum Design}
\newacronym{SD-TMSR}{SD-TMSR}{Single-fluid Double-zone Thorium-based Molten Salt Reactor}	
\newacronym{SFR}{SFR}{Sodium-Cooled Fast Reactor}
\newacronym{SINDAG}{SINDA{\textbackslash}G}{Systems Improved Numerical Differencing Analyzer $\backslash$ Gaski}
\newacronym{SKB}{SKB}{Svensk K\"{a}rnbr\"{a}nslehantering AB}
\newacronym{SNF}{SNF}{spent nuclear fuel}
\newacronym{SNL}{SNL}{Sandia National Laboratory}
\newacronym{STC}{STC}{specific temperature change}
\newacronym{SUPG}{SUPG}{Streamline-Upwind/Petrov-Galerkin}
\newacronym{SWF}{SWF}{Separations and Waste Forms}
\newacronym{SWU}{SWU}{Separative Work Unit}
\newacronym{TRIGA}{TRIGA}{Training Research Isotope General Atomic}
\newacronym{TRISO}{TRISO}{Tristructural Isotropic}
\newacronym{TSM}{TSM}{Total System Model}
\newacronym{TSPA}{TSPA}{Total System Performance Assessment for the Yucca Mountain License Application}
\newacronym{ThOX}{ThOX}{thorium oxide}
\newacronym{UFD}{UFD}{Used Fuel Disposition}
\newacronym{UML}{UML}{Unified Modeling Language}
\newacronym{UOX}{UOX}{uranium oxide}
\newacronym{UQ}{UQ}{uncertainty quantification}
\newacronym{US}{US}{United States}
\newacronym{UW}{UW}{University of Wisconsin}
\newacronym{VISION}{VISION}{the Verifiable Fuel Cycle Simulation Model}
\newacronym{VVER}{VVER}{Voda-Vodyanoi Energetichesky Reaktor (Russian Pressurized Water Reactor)}
\newacronym{VV}{V\&V}{verification and validation}
\newacronym{WIPP}{WIPP}{Waste Isolation Pilot Plant}
\newacronym{YMR}{YMR}{Yucca Mountain Repository Site}
\newacronym{CNRS}{CNRS}{National Center for Scientific Research}		
\newacronym{CRAM}{CRAM}{Chebyshev Rational Approximation Method}
\newacronym{DT}{DT}{Doubling Time}		
\newacronym{Euratom}{Euratom}{European Atomic Energy Community}
\newacronym{FPs}{FPs}{Fission Products} 
\newacronym{HM}{HM}{Heavy Metal}
\newacronym{MAs}{MAs}{Minor Actinides}
\newacronym{OpenMP}{OpenMP}{Open Multi-Processing}
\newacronym{TCR}{TCR}{Temperature Coefficient of Reactivity}
\newacronym{3D}{3D}{Three Dimensions}			
\newacronym{TS-MSR}{TS-MSR}{Thermal-Spectrum Molten Salt Reactor}
\newacronym{FS-MSR}{FS-MSR}{Fast-Spectrum Molten Salt Reactor}
\newacronym{EVOL}{EVOL}{Evaluation and Viability of Liquid Fuel Fast Reactor System}
\newacronym{TMSR}{TMSR}{Thorium Molten Salt Reactor}

\makeglossaries

%\journal{Annals of Nuclear Energy}

\begin{document}

%\begin{frontmatter}

% Title, authors and addresses

% use the tnoteref command within \title for footnotes;
% use the tnotetext command for the associated footnote;
% use the fnref command within \author or \address for footnotes;
% use the fntext command for the associated footnote;
% use the corref command within \author for corresponding author footnotes;
% use the cortext command for the associated footnote;
% use the ead command for the email address,
% and the form \ead[url] for the home page:
%
% \title{Title\tnoteref{label1}}
% \tnotetext[label1]{}
% \author{Name\corref{cor1}\fnref{label2}}
% \ead{email address}
% \ead[url]{home page}
% \fntext[label2]{}
% \cortext[cor1]{}
% \address{Address\fnref{label3}}
% \fntext[label3]{}

\title{Strategies for thorium fuel cycle transition in the SD-TMSR\\
\large Response to Review Comments}
\author{O. Ashraf, Andrei Rykhlevskii, G. V. Tikhomirov, Kathryn D. Huff}

% use optional labels to link authors explicitly to addresses:
% \author[label1,label2]{<author name>}
% \address[label1]{<address>}
% \address[label2]{<address>}


%\author[uiuc]{Kathryn Huff}
%        \ead{kdhuff@illinois.edu}
%  \address[uiuc]{Department of Nuclear, Plasma, and Radiological Engineering,
%        118 Talbot Laboratory, MC 234, Universicy of Illinois at
%        Urbana-Champaign, Urbana, IL 61801}
%
% \end{frontmatter}
\maketitle
\section*{Review General Response}
We would like to thank the reviewers for their detailed assessment of
this paper. Your suggestions, clarifications, and comments have resulted in 
changes which certainly improved the paper.


\begin{questions}
        \section*{Reviewer 1}

        \question This work investigates the fuel cycle and neutronics performance of the Single-fluid Double-zone Thorium-based Molten Salt Reactor (SD-TMSR) with different fissile material based on an extended SERPENT-2 code. The whole paper states clearly.

        \begin{solution}
                Thank you very much for these comments. We appreciate your detailed review, which has certainly improved the paper. The manuscript has been enhanced and more detail is described regarding these changes in the specific comment responses below.
        \end{solution}

        %---------------------------------------------------------------------

        \question In the SD-TMSR burnup calculation, how are the depletion regions and burnup time step defined?
        \begin{solution}
        		  
        		  Thank you for the excellent point.
        		  
        		  About depletion regions definition: we used depletion mode 1 in SERPENT-2 (i.e. burn 1), this means that the fuel material is not divided and all occurrences of it are treated as a single depletion zone. The fuel is liquid -it well- mixed all the time due to turbulent flow and mixing in the pump.
        		  
        		  About burnup time step definition: it is time interval given in days. During burnup time step, the SD-TMSR core was maintained critical and total fuel mass was almost constant (dm $\leq$ 0.1\%). We started our calculations by 30 days' time step but we do not find significant changes in $k_{eff}$ when we increased the time step to 365 days. The maximum change in $k_{eff}$ was $\leq$ 0.02\%), therefore we adopted 365-day as a time step for all simulation in this work.
        		
        		  We changed the paragraph as follows:\\
        		  The fuel is liquid -it well- mixed all the time due to turbulent flow and mixing in the pump. We used depletion mode 1 in SERPENT-2 (i.e. burn 1), this means that the fuel material is not divided and all occurrences of it are treated as a single depletion zone.
        		  
        		  We changed the paragraph as follows:\\
        		  During the depletion step, the core was maintained critical and total fuel mass was almost constant (dm $\leq$ 0.1\%). To determine the appropriate depletion step size, we conducted a time step refinement study. We started with $\Delta t=30$days and gradually increased the depletion time step until the error in $k_{eff}$ became significant. For $\Delta t=1$year the maximum change in $k_{eff}$ was $\leq$ 0.02\%), therefore we adopted a constant, 365-day-long time step for all simulation in this work.
        		    
        \end{solution}

        %---------------------------------------------------------------------
        \question During the molten salt reactor operation, part of fuel salt flows outside of reactor core then the fission reaction is not happened in external loop, however the decay reactions happens everywhere. In other words, during one burnup step, the fission reactions occur in only part of the time (depends on the ratio of time taken in fuel salt passing reactor core and external loop), and nuclides' decay or other reactions occur in the whole burnup step time. Have you considered this in SERPENT calculation?  
        \begin{solution}
		          
		          Thank you for your insightful comment. We used total fuel salt inventory in the primary loop to define the fuel material in the SERPENT input. Moreover, we accordingly adjusted power density in SERPENT burnup calculations because fission takes place only in the core region.
		          
		          We changed the paragraph as follows:\\    
		          During the molten salt reactor operation, part of fuel salt flows outside of the reactor core then the fission reaction has not happened in the external loop, however, the decay reactions happen everywhere. We used total fuel salt inventory in the primary loop to define the fuel material in the SERPENT input. Moreover, we accordingly adjusted power density in SERPENT burnup calculations because fission takes place only in the core region.     
        \end{solution}

        %---------------------------------------------------------------------
	
        \section*{Reviewer 2}

        %---------------------------------------------------------------------
        \question Strategies for thorium fuel cycle transition by employing different startup fissile material and feed mechanisms were investigated for the single-fluid double-zoned thorium-based molten salt reactor (SD-TMSR) in this study, and some specific conclusions were obtained. While, realizing thorium fuel cycle transition by using different startup fissile material has already been well studied in the previous studies for different types of MSR, including SD-TMSR. The motivation (e.g., improvement from previous studies) for conducting this study is not clear. In addition, some statements in the paper are inaccurate and more interpretations should be provided to explain some obtained results. Therefore, a major revision is required to improve the manuscript.
        
        \begin{solution}
                Thank you very much for these comments. We appreciate your detailed review, which has certainly improved the paper. The manuscript has been enhanced and more detail is described regarding these changes in the specific comment responses below.
        \end{solution}

        %---------------------------------------------------------------------

        \question SD-TMSR concept development does not correctly described in the section "introduction", where it states that the SD-TMSR was introduced by the Chinese Academy of Sciences. Actually, SD-TMSR concept was proposed by ORNL as early as in the 1960s, which was called Molten salt breeding reactor (MSBR) (see Ref.: Tech. Rep. ORNL-4541). In addition, the study of reference 7 which was cited to illustrate SD-TMSR concept actually adopted the single-flow single-zone MSR rather than SD-TMSR.
        
        \begin{solution}
                 Thank you for catching this. The whole statement has been modified to read:\\
                 The Single-fluid Double-zone Thorium-based Molten Salt Reactor (SD-TMSR) with a thermal power of 2,250
                 MW$_{th}$ was proposed for the first time by ORNL as early as in the 1960s, which was called Molten salt breeding reactor (MSBR) \cite{robertson_conceptual_1971}.
                 
                 Thank you for catching this. Reference 7 has been removed from the paragraph.
                 
        \end{solution}

        %---------------------------------------------------------------------

        \question The authors did not conduct a sufficient literature survey on the fuel cycle transition studies for the SD-TMSR concept. Cui. et al., have made an extensive study on fuel cycle transition for SD-TMSR by employing LEU and Pu as the startup fissile materials and two transition modes (in-core transition and ex-core transition) (see Ref.: Transition toward thorium fuel cycle in a molten salt reactor by using plutonium, NUCL SCI TECH (2017) 28:152; Possible scenarios for the transition to thorium fuel cycle in molten salt reactor by using enriched uranium. Progress in Nuclear Energy 104 (2018) 75-84). The state "For the SD-TMSR concept, simulating the fissile core startup and fuel cycle transition with other fissile materials (except U-233) has not studied before" in page 3 is therefore inaccurate.
        \begin{solution}
                Thank you for your helpful review. The state "For the SD-TMSR concept, simulating the fissile core startup and fuel cycle transition with other fissile materials (except$^{233}$U) has not studied before" in page 3, has been removed. 
                
                The following paragraphs have been added to the manuscript.\\
                Cui \emph{et al.} (2017-2018) analyzed the fuel transition from enriched $^{235}$U/Th and Pu/Th to $^{233}$U/Th for the MSBR-like design. Cui \emph{et al.} found that the fuel transition can be achieved by using: (1) enriched uranium with greater than 40\% enrichment, which rises the proliferation concerns; (2) Pu from LWR (burn-up of 60 GWd/t) spent nuclear fuel \cite{cui2017transition,cui2018possible}.
                
                The research effort described in \cite{cui2017transition} and
                \cite{cui2018possible} is most similar to the work
                presented in this paper. However, a few major differences
                are: (1) we used built-in truly continuous fuel depletion capabilities in
                SERPENT-2 \cite{aufiero2013extended} while Cui \emph{et al.} employed an in-house tool named MSR
                Reprocessing Sequence (MSR-RS) \cite{ZOU2015114} which couples with SCALE and employs a batch-wise approach (the burnup simulation stops at a given time and restarts with a new liquid fuel composition after removal of discarded materials and addition of fissile/fertile materials); (2) we investigated different initial loading cases (HALEU, Pu+HALEU, reactor-grade Pu, TRU and $^{233}$U) while Cui \emph{et al.} studied only uranium and Pu as the startup fissile materials; (3) we adopted the SD-TMSR core geometry optimized by Li \emph{et al.} \cite{li_optimization_2018} which is different from the MSBR-like model adopted by Cui \emph{et al.}; (4) Cui \emph{et al.} studied the high-enriched uranium (e$>$90) while we excluded it due to the unavailability of this material for use as an initial reactor fuel (non-proliferation issues) and because the performance of this material is expected to be similar to $^{233}$U; (5) Cui \emph{et al.} introduced two scenarios for thorium fuel cycle transition: a Breeding and Burning (B\&B) scenario and a Pre-breeding scenario while we adopting different approach by introducing two different feed mechanisms: thorium and non-thorium feed mechanism.
                
                In more detail, in current work, the radii of the fuel channels in the outer and inner zone are 5 and 3.5 cm, respectively, while in Cui \emph{et al.} work, all fuel channels have the same radius 2.03 cm, also, the side length of the graphite hexagonal prism is different in both models (7.5 instead of 10 cm in Cui model), total salt volume, density, and composition are different than those used in our study. In Cui's scenarios, the amount of Th is kept constant. $^{233}$U is fed only in (B\&B) scenario and extra fissile materials would have to be added into the core if the produced $^{233}$U is not enough to maintain the reactor criticality. In the Pre-breeding scenario, all the $^{233}$U produced from the decay of the extracted $^{233}$Pa stored outside the core until it reaches the required startup mass for a new reactor. Thus, the criticality of the core is maintained by refueling external fissile material. 
                In current work, using the thorium feed mechanism, we simultaneously feed Th and all or part of produced $^{233}$U from the \texttt{Pa-decay tank} into the core. The excess $^{233}$U (if exist) is stored outside the core. In the non-thorium feed mechanism, we continuously inject external heavy metals (HALEU, Pu mixed with HALEU, reactor-grade Pu, or TRU) and part of produced $^{233}$U from the \texttt{Pa-decay tank}. Here we feed a part of produced $^{233}$U because in some cases (e.g., TRU) refueling only external heavy metals failed to maintain the core criticality (limitation is dm$_{total fuel}$ $\leq$ 0.1\%). In the non-thorium feed mechanism, the amount of Th decreases during reactor operation while it is constant in the thorium feed mechanism.
                
        \end{solution}

        %---------------------------------------------------------------------
        

        \question More details should be provided to illustrate the heavy metal addition for the thorium feed mechanism and non-thorium feed mechanism in page 11. It is hard to understand why the mass of Th / external heavy metals (excluding Th) is approximately equal to the mass of extracted Pa.
        \begin{solution}
               Thank you for the comment. Actually, in page 11, we mention that the mass of Th / external heavy metals is approximately equal to the mass of extracted Fission products (FPs) and not Pa. Because in both feed mechanisms (thorium and non-thorium) we continuously added $^{233}$U and the mass of injected $^{233}$U is equal to the mass of extracted Pa ($^{233}$Pa decays to $^{233}$U after $\approx$ 27 d). To keep the total fuel mass almost constant, for \textbf{thorium feed mechanism}, the mass of injected Th must be equal to the mass of extracted FPs and for \textbf{non-thorium feed mechanism}, the mass of injected fuel (i.e., HALEU, Pu mixed with HALEU, reactor-grade Pu, or TRU) must be equal to the mass of extracted FPs.
              
               Moreover, we rewrite the paragraph in page 11 to address this thought:\\
               \begin{enumerate}
               	\item The simulation starts without injecting refueling materials (i.e. only removing FPs and Pa).
               	\item After the first depletion calculation step, we check the total mass density of FPs and Pa in the \texttt{FP-waste tank} and \texttt{Pa-decay tank}, respectively.
               	\item A simple calculation yields the amount of heavy metal that must be added during this cycle:
               	\subitem 3a. \textbf{Thorium feed mechanism}: mass of injected $^{233}$U $\approx$ mass of extracted Pa and mass of injected Th $\approx$ mass of extracted FPs to maintain the total fuel mass constant.
               	\subitem 3b. \textbf{Non-thorium feed mechanism}: mass of injected $^{233}$U $\approx$ mass of extracted Pa and mass of injected fuel (i.e., HALEU, Pu mixed with HALEU, reactor-grade Pu, or TRU) $\approx$ mass of extracted FPs to maintain the total fuel mass constant.
               	\item Dividing this mass by time and inventory of refueling material gives the corresponding feed constant (1/s).
               \end{enumerate}
               In both feed mechanisms (thorium and non-thorium) we continuously added $^{233}$U and the mass of injected $^{233}$U is equal to the mass of extracted Pa ($^{233}$Pa decays to $^{233}$U after $\approx$ 27 d). To keep the total fuel mass almost constant, for \textbf{thorium feed mechanism}, the mass of injected Th must be equal to the mass of extracted FPs and for \textbf{non-thorium feed mechanism}, the mass of injected fuel (i.e., HALEU, Pu mixed with HALEU, reactor-grade Pu, or TRU) must be equal to the mass of extracted FPs.
               
        \end{solution}

        %--------------------------------------------------------------------

        \question Phenomenons presented in Figure 3 that K-eff for the case of Pu+HALEU firstly increases and then decreases, and the time to become subcritical for the case of reactor-grade Pu is longest (except the case of U-233) should be explained.
        \begin{solution}
                 Thank you for your insightful comment. Phenomenons presented in Figure 3 have been described in the manuscript.
                 
                 The following paragraph has been added to the manuscript.\\
                 For Pu+HALEU case, $k_{eff}$ firstly increases during the first $\approx$ $3$ EFPY and then gradually decreases. This is due to the conversion of $^{238}$U from HALEU into fissile plutonium, which often produces the majority of the neutrons until the transition to Th/$^{233}$U fuel cycle. Figure~\ref{fig:diffPuAll811} shows the variation in fissile plutonium isotopes ($^{239}$Pu , $^{241}$Pu, and $^{243}$Pu) inventory during reactor operation. For TRU and reactor-grade Pu, the fissile plutonium isotopes are gradually depleted, during operating (the fissile material is dominated by Pu in these cases). For HALEU case, the fissile material is dominated by $^{235}$U, however, $^{238}$U is converted into fissile plutonium. Therefore, the mass of fissile plutonium increases during operating but its value still relatively low. For Pu+HALEU case, the mass of the fissile plutonium isotopes increases within the first $\approx$ $3$ EFPY and then decreases. The reactivity from the produced Pu is much greater than the depletion of the initial reactivity (due to the depletion of initial fissile isotopes), consequently, $k_{eff}$ increases during the first $\approx$ $3$ years and then decreases; the amount of $^{233}$U generated in the Pu+HALEU case is insufficient to maintain the reactor criticality and counteract neutron absorption in non-fissile isotopes.
                 
                 The following paragraph has been added to the manuscript.\\
                 Figure~\ref{fig:keff1} shows that the decrease in $k_{eff}$ for reactor-grade Pu case occurs at the much later operating time than for other cases because the neutron spectrum in the reactor-grade Pu initial core is hardened; more $^{232}$Th is being converted to $^{233}$U. Nevertheless, this amount of the $^{233}$U is not enough to counteract the neutron absorption in the non-fissile Pu isotopes after much of the $^{239}$Pu and $^{241}$Pu is depleted.
                 
        \end{solution}

        %--------------------------------------------------------------------
        \question In page 19, why the Pa extraction time was selected to be 30s since Pa is neither the gaseous FPs nor the non-dissolved metals?
        \begin{solution}
                  Thank you for the question. Specifically, Pa is removed from the fuel by chemical reprocessing into the \texttt{Pa-decay tank} to decay and produce $^{233}$U. The target is to maintain the core criticality for a long period of time without any external feed of $^{233}$U. Figure~\ref{fig:Keff_U2332} shows how Reprocessing Period (RP) affects the $k_{eff}$ and amount of $^{233}$U in the \texttt{Pa-decay tank} for the TRU-started SD-TMSR. The fast extraction of Pa from the core (RP = $30$ $s$) leads to long operating time. Also, a relatively large amount of $^{233}$U can be obtained when we applied RP = $30$ $s$ against $10.59$ $d$. The RP equal to $10.59$ $d$ is not suitable for TRU and reactor-grade Pu cases because additional $^{233}$U was required to supply the core after startup. Therefore, we selected the RP = $30$ $s$. This produces $^{233}$U in the \texttt{Pa-decay tank} barely enough to operate the TRU-started SD-TMSR for a long period of time ($\approx$ 40 years). For reactor-grade Pu case, the produced $^{233}$U is sufficient to maintain the core criticality within the 60-years of operation (see Figure~\ref{fig:keff2}).
                  
                  The following paragraph has been added to the manuscript.\\
                  The target is to maintain the core criticality for a long period of time without any external feed of $^{233}$U. Figure~\ref{fig:Keff_U2332} shows how Reprocessing Period (RP) affects the $k_{eff}$ and amount of $^{233}$U in the \texttt{Pa-decay tank} for the TRU-started SD-TMSR. The fast extraction of Pa from the core (RP = $30$ $s$) leads to long operating time. Also, a relatively large amount of $^{233}$U can be obtained when we applied RP = $30$ $s$ against $10.59$ $d$. The RP equal to $10.59$ $d$ is not suitable for TRU and reactor-grade Pu cases because additional $^{233}$U was required to supply the core after startup. Therefore, we selected the RP = $30$ $s$. This produces $^{233}$U in the \texttt{Pa-decay tank} barely enough to operate the TRU-started SD-TMSR for a long period of time ($\approx$ 40 years). For reactor-grade Pu case, the produced $^{233}$U is sufficient to maintain the core criticality within the 60-years of operation (see Figure~\ref{fig:keff2}).
        \end{solution}

        %--------------------------------------------------------------------
        \question For Figure 9, it is inaccurate to state that the mass of U-233 reaches equilibrium after ~30 years, apparently from the figure, we can observe that the mass of U-233 for three of the cases still increase/decrease after 30 years.
        
        \begin{solution}
                Thank you for the excellent point. The statement has been removed from the text.
                Moreover, we rewrite the paragraph to address this thought:\\
                Figure~\ref{fig:U233CC} demonstrates the mass of $^{233}$U in the fuel salt for the $^{233}$U, reactor-grade Pu, and TRU cases. For $^{233}$U case, the mass of $^{233}$U increases with operating time and reaches $\approx$ $1.7$ $t$ at the EOL. For reactor-grade Pu and TRU cases, the mass of $^{233}$U firstly, increases and after $\approx$ $35$ and $45$ years for Pu and TRU cases, respectively, decreases. The neutron spectrum shift during operating affects the inventory of $^{233}$U in the fuel salt. Hard neutron spectrum converts more $^{233}$U from $^{232}$Th. For the reactor-grade Pu and TRU cases, during operating, the fissile Pu is depleted and the $^{233}$U becomes the major fissile isotope; the neutron spectrum softens (low conversation of $^{233}$U from $^{232}$Th). However, for $^{233}$U case, the neutron spectrum is hardened at EOL due to the accumulation of Pu isotopes.                
                
        \end{solution}

        %--------------------------------------------------------------------
        \question Regarding to Figure 10, reasons should be provided to explain why the time achieving Pu solubility limit for the case of TRU is shorter than that of the case of reactor-grade Pu.
        \begin{solution}
                 This has now been clarified in the text.
                 We rewrite the paragraph to address this thought:\\
                 Notably, for TRU case, the time achieving Pu solubility limit is shorter than that for reactor-grade Pu case. The reason for this is the relatively high concentration of Pu in the initial heavy metal for the TRU-started SD-TMSR ($\approx$ $29.26$ \% compared with $14.31$ \% for Pu-started SD-TMSR, see Table 5).
                 
        \end{solution}

        %--------------------------------------------------------------------
        \question It is hard to understand that the neutron spectrum of TRU case at BOL is softer than U-233 case since TRU has a greater amount of thermal neutron absorbers and the fissile material is dominated by Pu which would generate hard neutron spectrum.
        
        \begin{solution}
                 We appreciate your detailed review. 
                  
                 The following paragraph has been added to the manuscript.\\
                 The neutron spectrum of TRU case at BOL is softer than $^{233}$U case because $^{232}$Th inventory is much lower for the TRU case: $54.4$ $t$ instead of $76.9$ $t$ for the $^{233}$U. $^{232}$Th has a very high absorption cross section in the thermal region ($\approx10^2 $b). Moreover, $^{232}$Th has resonance region between $10^{-5}$ and $10^{-3}$ which justify relatively low energy of neutrons for the TRU case in this energy range (see Figure~\ref{fig:spectrumFLUX110vC}). 
                 
        \end{solution}

        %--------------------------------------------------------------------
        \question More details should be provided to explain the radial neutron flux distribution for the cases of Pu reactor-grade and TRU presented in Fig.13 and 14. For both the radial fast and thermal neutron flux change from BOL to EOL, why it is much larger for Pu reactor-grade case compared with TRU case. In addition, the state "In contrast, for the reactor-grade Pu and TRU cases, significant changes are observed for thermal neutron flux in the outer core zone and reactor" is inaccurate, apparently from Figure 14, it can be seen that the Pu reactor-grade (initial) case has a small change in the outer core zone and reflector.
        \begin{solution}
                Thanks  
        \end{solution}

        %--------------------------------------------------------------------
        \question Regarding to the FTC for three different cases, reasons should be provided to explain why spectrum softening will positively affects the FTC since generally harder neutron spectrum would benefit FTC.
        \begin{solution}
                Thank  
        \end{solution}

        %--------------------------------------------------------------------
        \question It seems unclear why (what is the purpose) to conduct the study of six factor analysis.
        \begin{solution}
                 Thanks  
        \end{solution}

        %--------------------------------------------------------------------
        \question Making the full definition for abbreviation through the whole paper is not necessary, such as SD-TMSR, first time is enough.
        
        \begin{solution}
                Thank you for the information. The full definition for abbreviation through the whole paper has been removed as requested.
        \end{solution}

        %--------------------------------------------------------------------

        
        
\end{questions}
\bibliographystyle{elsarticle-num}
\bibliography{../2019-Scenarios}
\end{document}

%
% End of file `elsarticle-template-num.tex'.
