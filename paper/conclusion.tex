\section{Conclusion} \label{Conclusion}

Five different types of initial fissile loadings are considered to investigate 
transitioning to the thorium fuel cycle in the SD-TMSR. We adopted two 
different feed mechanisms: thorium feed mechanism and non-thorium feed 
mechanism. Lifetime-long depletion for the whole-core SD-TMSR model was 
performed with reactor-grade Pu, TRU, and $^{233}$U as initial fissile 
materials. Additionally, the dynamics of the effective multiplication factor 
$k_{eff}$, major isotopes mass, neutron energy spectrum, and essential safety 
parameters have been investigated. 

Results demonstrate that continuous flow of reactor-grade Pu allows the 
transition to the thorium fuel cycle in a relatively short time ($\approx$ 
$4.5$ years) compared to $26$ years for Th/$^{233}$U startup fuel. 
Meanwhile, using \gls{TRU} as initial fissile materials shows the possibility 
of operating the SD-TMSR for an extended time ($\approx$ $40$ years) 
without any external feed of $^{233}$U. Notably, the Pu molar fraction (mole\%) in 
fuel salt was calculated and found to be below the solubility limit. 

The neutron energy spectrum shift during the reactor operation 
for the Pu and TRU cases is different from the $^{233}$U fueling scenario. 
The spectrum hardens for the $^{233}$U initial fissile isotope during 
operation, but softens for the Pu and TRU cases. Notably, the most 
significant neutron energy spectrum shift was obtained for reactor-grade Pu 
startup loading. 

We compared the operational and safety parameters of the \gls{SD-TMSR} for all 
three startup fuels at both initial and equilibrium states. The total 
temperature coefficient of reactivity is negative and relatively large in all 
cases. For the TRU case, the coefficient remained almost constant during 
operation: $-4.79\pm0.12$ $pcm/K$ and $-4.76\pm0.11$ $pcm/K$ for the initial 
and equilibrium states, respectively. For reactor-grade Pu, the coefficient 
absolute value decreased from $-6.54\pm0.16$ $pcm/K$ to $-4.79\pm0.12$ during 
60 years of operation. Finally, the six factors evolution during the operation 
were calculated for all three cases, and these parameters can be used to 
design the reactivity control system of the \gls{SD-TMSR}.

\section{Future work}
The authors intend to verify obtained results using another tool: batch-wise 
code SaltProc \cite{rykhlevskii_arfc/saltproc_2018,rykhlevskii_milestone_2019}.
In further simulations, we intend to take into account the delayed 
neutron precursor drift. The \gls{SD-TMSR} reactivity control system has not 
been introduced in the literature yet. Thus, the control rods design and 
configuration might be suggested in the nearest future.

An additional area to explore is the accident safety analysis which 
requires high-fidelity multi-physics model of the \gls{SD-TMSR} with the 
coupled neutronics/thermal-hydraulics software, Moltres 
\cite{lindsay_introduction_2018}. The full-core SERPENT-2 model of the 
\gls{SD-TMSR} and equilibrium fuel salt compositions, obtained in this work, 
would be employed to generate problem-oriented nuclear data libraries for 
Moltres. The ultimate goal of this effort is to develop a 
fast-running computational model for studying the dynamics behavior of generic 
\glspl{MSR}, performing safety analysis for different accident scenarios and 
optimizing design of various reactor concepts.

\section{Declaration of Competing Interest}

The authors declare that they have no known competing financial interests or personal relationships that could have appeared to influence the work reported in this paper.
