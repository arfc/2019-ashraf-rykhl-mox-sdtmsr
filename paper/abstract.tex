\begin{abstract}
Liquid-fueled \gls{MSR} systems represent advances in safety, economics, and sustainability. The \gls{MSR} has been designed to operate with a Th/$^{233}$U fuel cycle with $^{233}$U used as startup fissile material. Since $^{233}$U does not exist in nature, we must examine other available fissile materials to start up these reactor concepts. This work investigates the fuel cycle and neutronics performance of the \gls{SD-TMSR} with different fissile material loadings at startup: \gls{HALEU} (19.79\%), Pu mixed with \gls{HALEU} (19.79\%), reactor-grade Pu (a mixture of Pu isotopes chemically extracted from \gls{PWR} \gls{SNF} with $33$ $GWd/tHM$ burnup), \gls{TRU} from \gls{LWR} \gls{SNF}, and $^{233}$U.
The \gls{MSR} burnup routine provided by SERPENT-2 is used to simulate the online reprocessing and refueling in the \gls{SD-TMSR}. The effective multiplication factor, fuel salt composition evolution, and net production of $^{233}$U are studied in the present work. Additionally, the neutron spectrum shift during the reactor operation is calculated. The results show that the continuous flow of reactor-grade Pu helps transition to the thorium fuel cycle within a relatively short time ($\approx$ $4.5$ years) compared to $26$ years for $^{233}$U startup fuel. Finally, using \gls{TRU} as the initial fuel materials offers the possibility of operating the SD-TMSR for an extended period of time ($\approx$ $40$ years) without any external feed of $^{233}$U.
\end{abstract}
