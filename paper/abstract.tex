\begin{abstract}
Liquid-fueled \gls{MSR} systems represent advances in safety, economics, sustainability, and proliferation resistance. The \gls{MSR} has been designed to operate in Th/$^{233}$U fuel cycle with $^{233}$U used as startup fissile material. Since $^{233}$U does not exist in nature, We must examine other available fissile materials. This work investigates the fuel cycle and neutronics performance of the \gls{SD-TMSR} with different fissile material loadings at startup: \gls{LEU} (19.79\%), Pu mixed with \gls{LEU} (19.79\%), reactor-grade Pu (a mixture of Pu isotopes chemically extracted from \gls{PWR} \gls{SNF} with $33$ $GWd/tHM$ burnup), \gls{TRU} from \gls{LWR} \gls{SNF}, and $^{233}$U.
The \gls{MSR} burnup routine provided by SERPENT-2 is used to simulate the online reprocessing and refueling in the \gls{SD-TMSR}. The effective multiplication factor, fuel salt composition evolution and net production of $^{233}$U are studied in the present work. Additionally, the neutron spectrum shift during the reactor operation is calculated. The results show that the continuous flow of reactor-grade Pu helps transition to the thorium fuel cycle within a relatively short time ($\approx$ $4.5$ years) compared to $26$ years for $^{233}$U startup fuel. Finally, using \gls{TRU} as initial fissile materials offers the possibility of operating the SD-TMSR for an extended period of time ($\approx$ $40$ years) without any external feed of $^{233}$U.
\end{abstract}



%The SD-TMSR (2,250 MW$_{th}$) is a Single-fluid Double-zone Thorium-based Molten Salt Reactor. The active core of the SD-TMSR is divided into the inner zone (486 fuel tubes) and the outer zone (522 fuel tubes) to improve the Th-U breeding performance. The lack of computational reactor analysis software that deal with MSR cores prevents improvement. This work adopted the SERPENT-2 Monte Carlo code to analyze the whole core model of the SD-TMSR. Built-in SERPENT-2 capabilities simulated online reprocessing and refueling and calculated the multiplication factor and Breeding Ratio (BR). We found that the molten salt Temperature Coefficient of Reactivity $(\alpha$$_{T}$) was negative for initial and equilibrium states. This study investigated the variation of the multiplication factor, BR, and build-up of important nuclides in the core as a function of burn-up time. Under online reprocessing and refueling, we studied the variation of the reactivity during 60 years of reactor operation. The present work has introduced a feasible reprocessing scheme. In addition, it has investigated the neutron flux energy spectrum at initial and equilibrium state.
%In conclusion, the full core of SD-TMSR was modeled and the net production of $^{233}$U was found to be positive.


%Therefore, \gls{MSR} has been selected as one of the promising reactors by the \gls{GIF}.
