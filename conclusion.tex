\section{Conclusion}

In the present paper, five different types of initial fissile materials have been studied for transitioning to thorium fuel cycle in the \gls{SD-TMSR}. The molar composition of startup fuel for all five cases is listed in Table~\ref{tab:table4}, as well, the inventories in Table~\ref{tab:table5}. We adopted two different feed mechanisms; thorium feed mechanism and non-thorium feed mechanism. The whole-core of the SD-TMSR was simulated with Pu reactor-grade, TRU, and $^{233}$U as initial fissile materials. Besides, the variation of the effective multiplication factor $k_{eff}$, inventory, and other neutronic parameters have been investigated. Results demonstrated that continuous flow of Pu reactor-grade helps in transition to thorium fuel cycle within a relatively short time ($\approx$ $4.5$ $years$) compared to $26$ $years$ for Th/$^{233}$U startup fuel. Meanwhile, using \gls{TRU} as initial fissile materials shows the possibility of operating the SD-TMSR for a long period of time ($\approx$ $40$ $years$) without any external feed of $^{233}$U. in addition, the Pu proportion in fuel salt has been calculated and found to be below the solubility limit. Finally, the neutron flux spectrum for the three selected cases has been described.

\section{Future work}

%The authors intend to compare the present results and results from SaltProc batch-wise code \cite{rykhlevskii2019modeling} in future work. Moreover, they will consider the delayed neutron precursor drift. Future work will also consider the benchmark study based on other codes and models. The authors intend to conduct a sensitivity study to evaluate the impact of the radial reflector material and its thickness. In addition, they will study the impact of \gls{FPs} removal (i.e. only reprocessing without refueling) on reactivity and neutron economy.

\section{Conflict of interest}

The authors declare no conflict of interest. 
